%%%%%%%%%%%%%%%%%%%%%%%%%%%%%%%%%%%%%%%%%%%%%%%%%%%%%%%%%%%%%%%%%%%%%%%%%%%
% Chicago 17 main.text file for Academic Papers.
%%%%%%%%%%%%%%%%%%%%%%%%%%%%%%%%%%%%%%%%%%%%%%%%%%%%%%%%%%%%%%%%%%%%%%%%%%%

%%%%%%%%%%%%%%%%%%%%%%%%%%%%%%%%%%%%%%%%%%%%%%%%%%%%%%%%%%%%%%%%%%%%%%%%%%%
% main.tex Overview:
%   \documentclass{chicago17} % Loads definitions & packages 
%   \begin{document}
%   \printtitlepage     % Title page
%   \papercontents      % Optional Table of Contents
%   \content            % Your content goes here!
%   \printbibliography  % Prints biography
%   \end{document}
%%%%%%%%%%%%%%%%%%%%%%%%%%%%%%%%%%%%%%%%%%%%%%%%%%%%%%%%%%%%%%%%%%%%%%%%%%%

%%%%%%%%%%%%%%% Basic Configuration %%%%%%%%%%%%%%% 
% All the packages and configuration are loaded in the chicago17.cls file.
% For cleanness and simplicty, almost everything happens there!
\documentclass{chicago17}

\begin{document}

%%%%%%%%%%%%%%%  Title Page %%%%%%%%%%%%%%% 
% Set your variables for the title page:
\newcommand{\Title}{[PAPER TITLE (CAPS)]}
\newcommand{\Subtitle}{[SUB TITLE]}
\newcommand{\Course}{[Course ID and Name]}
\newcommand{\Author}{[Author Name]}
\newcommand{\Date}{\today}  % Feel free to remove \today and your own date.

% Create the title page using the above variables:
\printtitlepage

%%%%%%%%%%%%%%%  Table of Contents %%%%%%%%%%%%%%% 
% Uncomment the following to get a table of contents:
%\papercontents

%%%%%%%%%%%%%%%  Main Content %%%%%%%%%%%%%%%
\content

% Your paper goes here:

\section{Section (Level 1): Formatting \& Content Tests}

\subsection{Subsection (Level 2): Chicago Footnotes}
If you wish to do a Chicago style footnote, use \texttt{\textbackslash autocite}.  Here is an example: \enquote{good night moon.}\autocite[7]{brown_goodnight_moon} 

\subsection{Subsection (Level 2): Chicago Quotes}
Use \texttt{\textbackslash enquote} or other standard \LaTeX command if you are doing a short quote.  You'll want the quotation marks.  Here is an example (with footnote), \enquote{see Spot run!}\autocite[8]{gray_fun_with_dick_jane}  

If you wish to do a longer quote, use \texttt{\textbackslash chicagoblock}. Here is an example quoting from Purdue OWL recommendations for Chicago 17:
\chicagoblock{Text should be consistently double-spaced, except for block quotations, notes, bibliography entries, table titles, and figure captions. For block quotations, which are also called extracts.  Text should be consistently double-spaced, except for block quotations, notes, bibliography entries, table titles, and figure captions. For block quotations, which are also called extracts A prose quotation of five or more lines, or more than 100 words, should be blocked.
CMOS recommends blocking two or more lines of poetry. A blocked quotation does not get enclosed in quotation marks. A blocked quotation must always begin a new line.  Blocked quotations should be indented with the word processor’s indention tool.\footnote{Formatting note, I have removed the bullet points, opting for a block of text. \autocite{purdueowl_chicago_format}.}}

\subsubsection{Subsubsection (Level 3): Language Tests}
Here is some Greek:\textgreek{Ἐν ἀρχῇ ἦν ὁ λόγος}.\footnote{Jn. 1:1}  Here is some Hebrew:  \cjRL{bd'} 

Best wishes with your paper!  Enjoy the process and have fun!

\pagebreak

%%%%%%%%%%%%%%%  Bibliography %%%%%%%%%%%%%%% 
% Command to Create Bibliography
\printbibliography

\end{document}
